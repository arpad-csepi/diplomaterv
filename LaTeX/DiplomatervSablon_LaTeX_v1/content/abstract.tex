\pagenumbering{roman}
\setcounter{page}{1}

\selecthungarian

%----------------------------------------------------------------------------
% Abstract in Hungarian
%----------------------------------------------------------------------------
\chapter*{Kivonat}\addcontentsline{toc}{chapter}{Kivonat}

Ennek a szakdolgozatnak a célja, hogy ismertesse a mai világ nélkülözhetetlen felhőalapú számítástechnikai rendszerek automatizált telepítését, menedzselését és működését a Kubernetes szoftver segítségével, illetve ingyenes és nyílt forráskodú megoldást kínáljon ezekre. 

Az automatizált telepítés és klaszterek összekapcsolásának segítésére készítettem egy konzolos applikációt Go programozási nyelv segítségével. Ez a felhasználóbarát megoldás elősegíti, hogy több a téma iránt érdeklődő embereknek nyújton egy eszközt, ami könnyen használható, átlátható és módosítható.

A szakdolgozatom első szakaszában ismertetem azoknak a technológiáknak a hátterét, melyekre a program működésének megértéséhez szükség van.

Ezután bemutatom a programom tervezésének folyamatát, majd részletesen leírom az elkészített program szerkezeti felépítését és működését. Itt rávilágítok a program megvalósított funkcióira, a moduláris felépítésére, kiemelve a modulok legfontosabb részleteit. Bemutatom a tesztelési módszereket és dokumentálom az eredményeit.

A szakdolgozat végén összegzem a tanulságokat, elért eredményeket és további lehetséges fejlesztési lehetőségeket vázolok fel.

\vfill
\selectenglish


%----------------------------------------------------------------------------
% Abstract in English
%----------------------------------------------------------------------------
\chapter*{Abstract}\addcontentsline{toc}{chapter}{Abstract}

The purpose of this thesis is to describe the automated deployment, management and operation of cloud computing systems that are essential in today's world using Kubernetes software, and to provide a free and open-source solution for these. 

To help with automated deployment and cluster attaching, I created a console application using the Go programming language. This user-friendly solution helps to provide several people interested in the topic with a tool that is easy to use, transparent and modifiable.

In the first section of my thesis, I will describe the background of the technologies that are needed to understand how the program works.

I will then describe the process of designing my program, and then describe in detail the structure and operation of the program I have created. Here I will highlight the implemented features of the program, its modular structure, highlighting the most important details of the modules. I will present the testing methods and document the results.

At the end of the thesis, I summarise the lessons learned, the results achieved and outline further possible improvements.

\vfill
\selectthesislanguage

\newcounter{romanPage}
\setcounter{romanPage}{\value{page}}
\stepcounter{romanPage}