%----------------------------------------------------------------------------
\chapter{\bevezetes}
%----------------------------------------------------------------------------

A mai rohanó világban a modern alkalmazások egyre összetettebbé válnak, így felosztjuk kisebb részekre.
Ezek a mikroszolgáltatások egy elosztott architektúrára épülnek, amelyek különböző platformokon és környezetekben futnak.
Ez a komplexitás olyan kihívásokat hoz magával, mint a szolgáltatások közötti kommunikáció kezelése, biztonságának biztosítása és az egyes szolgáltatások teljesítményének felügyelete.
Az Istio Service Mesh termék ezeket a kihívásokat segíti áthidalni olyan funkciókkal amelyek egyszerűsítik a mikroszolgáltatás-alapú alkalmazások és klaszterek kezelését.

Cisco által fejlesztett Istio operátor Calisti Service Mesh Manager (későbbiekben Calisti) termék része, mely felhasználóbarát telepítést, klaszter menedzselést és megfigyelést tesz lehetővé egy modern webes felületen. A Calisti több komponensből álló összetett szolgáltatás, így működésének megértése időigényes és megterhelő lehet a felhő technológia iránt érdeklődő egyéneknek.

A Calisti Service Mesh Manager fizetős termék mellé kínálok egy ingyenes és teljesen nyílt forráskódú alternatívát.
Az általam fejlesztett program ugyanazt az Istio operátort használva implementál bizonyos funkciókat, mellyel felhasználóbarát és automatizált lehetőségeket kínálok telepítésre, törlésre, és több klaszter összekapcsolására.
A célom ezzel a programmal, hogy több emberhez eljusson a mikroszolgáltatás alapú szoftveres megoldások és képesek legyenek további fejlesztésekkel bővíteni és testreszabni azt.

\newpage

A szakdolgozat fejezeteit ajánlott sorrendben elolvasni, mert a fejezetek támaszkodnak előző fejezetekben leírt információkra.
Ezek technológiák ismerete szükséges lesz későbbiekben a szakdolgozat feladatának megértéséhez.
A témakörök amikről szó lesz röviden összefoglalva:
\begin{itemize}
    \item Konténerek előnyeinek ismertetése mikroszolgáltatások üzemeltetésekor.
    \item A népszerű Docker platform megismertetése, konténerképek rétegeinek bemutatása.
    \item Kubernetes alkalmazéskezelő működésének illusztrálása, master-slave architektúra és a különböző típusú erőforrások leírásával.
    \item Kubernetes in Docker lehetőségeinek bemutatása.
    \item Helm csomagkezelő használatának kifejtése.
    \item Istio Service Mesh való mikroszolgáltatások menedzselésének megismertetése.
    \item Felhő telepítésének típusai és a szolgáltatások típusainak körbejárása.
    \item A Go programozási nyelv ismertetése.
    \item Verziókezelő rendszerek bemutatása.
    \item A GitHub Actions leírása példával illusztrálva.
    \item A fejlesztett program felépítésének tervezéséről, kivitelezéséről, teszteléséről, kritikájáról és további fejlesztési potenciáljairól való dokumentáció.
\end{itemize}