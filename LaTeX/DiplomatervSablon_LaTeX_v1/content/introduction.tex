%----------------------------------------------------------------------------
\chapter{\bevezetes}
%----------------------------------------------------------------------------

\paragraph{LaTeX minta ajánlása alapján}\mbox{}\smallskip

A bevezető tartalmazza a diplomaterv-kiírás elemzését, történelmi előzményeit, a feladat indokoltságát (a motiváció leírását), az eddigi megoldásokat, és ennek tükrében a hallgató megoldásának összefoglalását.

A bevezető szokás szerint a diplomaterv felépítésével záródik, azaz annak rövid leírásával, hogy melyik fejezet mivel foglalkozik.

\paragraph{TMIT GYIK ajánlása alapján}\mbox{}\smallskip

A bevezetés célja, hogy az olvasó válasz kapjon a következő kérdésekre:
\begin{itemize}
    \item hol helyezkedik el a téma a világban,
    \item mi a megoldandó feladat,
    \item mi teszi indokolttá a probléma kezelését,
    \item és a probléma megoldása mit tesz lehetővé a számunkra?
\end{itemize}

A fejezet végén, a diplomaterv felépítését foglald össze, minden fejezetről egy-egy bővített mondatot írj, hogy mivel fog foglalkozni a kérdéses szövegrész.

Annyira legyen részletes a bevezetés, hogy egy távközlési/informatikai alapműveltséggel rendelkező mérnök a bevezető alapján megértse, követni tudja a diplomafeladatod speciális problémáit. A legjobb teszt: add oda egy olyan évfolyamtársadnak, aki nem ismerős a diplomád témájában, és kérdezd meg, érti-e miről van szó!