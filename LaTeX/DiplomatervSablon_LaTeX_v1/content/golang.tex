%----------------------------------------------------------------------------
\chapter{\golang}
%----------------------------------------------------------------------------
\section{A programozási nyelv}
Minden programozási nyelv tükrözi alkotóinak programozási filozófiáját, amely gyakran jelentős mértékben reagál a korábbi nyelvek vélt hiányosságaira.
A Go projekt a Google számos szoftverrendszerével kapcsolatos csalódottságból született, amelyek a komplexitás robbanásszerű növekedésétől szenvedtek.
(Ez a probléma korántsem csak a Google-nál jelentkezik.)
Ahogy Rob Pike fogalmazott, a ''komplexitás multiplikatív'': egy probléma megoldása a rendszer egy részének bonyolultabbá tételével lassan, de biztosan növeli a többi rész komplexitását.
A funkciók, opciók és konfigurációk hozzáadására, valamint a kód gyors szállítására irányuló állandó nyomás miatt könnyű elhanyagolni az egyszerűséget, pedig hosszú távon az egyszerűség a jó szoftver kulcsa.
Az egyszerűség több munkát igényel a projekt elején, hogy egy ötletet a lényegére redukáljunk, és több fegyelmet a projekt teljes élettartama alatt, hogy megkülönböztessük a jó változtatásokat a rossz vagy káros változtatásoktól.
Elegendő erőfeszítéssel egy jó változtatás a Fred Brooks által ''koncepcionális integritásnak'' nevezett terv veszélyeztetése nélkül is megvalósítható, de egy rossz változtatás nem, és egy ártalmas változtatás az egyszerűséget felszínes rokonára, a kényelemre cseréli.
Csak a tervezés egyszerűsége révén maradhat egy rendszer stabil, biztonságos és koherens, miközben növekszik.
A Go projekt magában foglalja magát a nyelvet, az eszközöket és a szabványos könyvtárakat, és nem utolsósorban a radikális egyszerűség kulturális programját.

Mivel a Go egy nemrégiben kifejlesztett magas szintű nyelv, a visszatekintés előnyeit élvezheti, és az alapok jól sikerültek: van szemétgyűjtés, csomagrendszer, első osztályú függvények, lexikális hatókör, rendszerhívó interfész és megváltoztathatatlan karakterláncok, amelyekben a szöveg általában UTF-8 kódolású.
Viszonylag kevés funkcióval rendelkezik, és nem valószínű, hogy továbbiakkal bővül. Például nincsenek implicit numerikus konverziók, nincsenek konstruktorok vagy destruktorok, nincs operátor-túlterhelés, nincsenek alapértelmezett paraméterértékek, nincs öröklés, nincs generika, nincsenek kivételek, nincsenek makrók, nincsenek függvényjegyzetek és nincs szál-lokális tárolás.
A nyelv kiforrott és stabil, és garantálja a visszafelé kompatibilitást: a régebbi Go programok lefordíthatók és futtathatók a fordítóprogramok és szabványos könyvtárak újabb verzióival.

A Go elégséges típusrendszerrel rendelkezik ahhoz, hogy elkerülje a legtöbb gondatlan hibát, amely a dinamikus nyelvek programozóit sújtja, de egyszerűbb típusrendszerrel rendelkezik, mint a hasonló tipizált nyelvek.
Ez a megközelítés néha a ''tipizálatlan'' programozás elszigetelt zugait eredményezheti a típusok tágabb keretén belül, és a Go programozók nem mennek el olyan messzire, mint a C++ vagy a Haskell programozók, hogy a biztonsági tulajdonságokat típusalapú bizonyításként fejezzék ki.
A gyakorlatban azonban a Go a programozóknak a viszonylag erős típusrendszerek biztonsági és futásidőbeli teljesítménybeli előnyeinek nagy részét biztosítja, anélkül, hogy egy összetett típusrendszer terhei terhelnék őket.
A Go ösztönzi a kortárs számítógépes rendszerek tervezésének tudatosítását, különösen a lokalitás fontosságát.
A beépített adattípusai és a legtöbb könyvtári adatszerkezet úgy van kialakítva, hogy természetesen explicit inicializálás vagy implicit konstruktorok nélkül működjön, így viszonylag kevés memóriafoglalást és memóriaírást rejt a kód.

A Go összesített típusai (struct-ok és tömbök) közvetlenül tartják elemeiket, így kevesebb tárolást és kevesebb allokációt és mutató indirekciót igényelnek, mint az indirekt mezőket használó nyelvek.
Mivel a modern számítógép egy párhuzamos gép, a Go rendelkezik a CSP-n alapuló párhuzamossági funkciókkal, ahogy azt már korábban említettük.
A Go könnyű szálainak vagy goroutine-ainak változó méretű halmai kezdetben elég kicsik ahhoz, hogy egy goroutine létrehozása olcsó, és egymillió létrehozása praktikus.
A Go szabványos könyvtára, amelyet gyakran úgy jellemeznek, hogy "elemekkel együtt" érkezik, tiszta építőelemeket és API-kat biztosít az I/O, a szövegfeldolgozás, a grafika, a kriptográfia, a hálózatépítés és az elosztott alkalmazások számára, számos szabványos fájlformátum és protokoll támogatásával.
A könyvtárak és az eszközök széleskörűen használják a konvenciókat, hogy csökkentsék a konfiguráció és a magyarázat szükségességét, ezáltal egyszerűsítve a programlogikát, és a különböző Go programokat egymáshoz hasonlóbbá és ezáltal könnyebben tanulhatóvá téve.
A go eszközzel épített projektek csak fájl- és azonosítóneveket, valamint egy alkalmi speciális megjegyzést használnak a projekt összes könyvtárának, futtatható fájljának, tesztjének, tesztjének, benchmarkjának, példájának, platform-specifikus változatának és dokumentációjának meghatározásához; a Go forrás maga tartalmazza az építési specifikációt \cite{Alan15}.

\section{Csomagkezelő}
Egy szerény méretű program ma akár 10.000 függvényt is tartalmazhat.
Szerzőjének azonban csak néhányra kell gondolnia, és még kevesebbet kell megterveznie, mert a túlnyomó többséget mások írták, és csomagok révén újrafelhasználásra bocsátották.
A Go több mint 100 szabványos csomagot tartalmaz, amelyek a legtöbb alkalmazás alapját képezik.
A Go közösség, a csomagtervezés, -megosztás, -újrafelhasználás és -fejlesztés virágzó ökoszisztémája még sokkal többet publikált, és ezek kereshető indexe megtalálható a http://godoc.org oldalon.
A Go a go tool-t is tartalmazza, egy kifinomult, de egyszerűen használható parancsot a Go csomagok munkaterületeinek kezelésére.

Minden csomagrendszer célja, hogy a nagy programok tervezését és karbantartását praktikussá tegye azáltal, hogy a kapcsolódó funkciókat könnyen értelmezhető és módosítható egységekbe csoportosítja, függetlenül a program többi csomagjától.
Ez a modularitás lehetővé teszi, hogy a csomagokat különböző projektek megosszák és újra felhasználják, egy szervezeten belül terjesszék, vagy a nagyvilág számára elérhetővé tegyék.
Minden csomag egy külön névteret határoz meg, amely az azonosítóit foglalja magába.
Minden név egy adott csomaghoz kapcsolódik, így rövid, egyértelmű neveket választhatunk a leggyakrabban használt típusokhoz, függvényekhez stb. anélkül, hogy konfliktusokat okoznánk a program más részeivel.
A csomagok kapszulázást is biztosítanak azáltal, hogy szabályozzák, mely nevek láthatók vagy exportálhatók a csomagon kívülre.
A csomagtagok láthatóságának korlátozása elrejti a segédfüggvényeket és típusokat a csomag API-ja mögé, lehetővé téve a csomag karbantartójának, hogy az implementáció megváltoztatásával biztos lehessen abban, hogy a csomagon kívüli kódot ez nem érinti.
A láthatóság korlátozása a változókat is elrejti, így az ügyfelek csak olyan exportált függvényeken keresztül érhetik el és frissíthetik őket, amelyek megőrzik a belső invariánsokat, vagy kölcsönös kizárást kényszerítenek ki egy párhuzamos programban.
Amikor megváltoztatunk egy fájlt, újra kell fordítanunk a fájl csomagját és potenciálisan az összes olyan csomagot, amely függ tőle.
A Go fordítása jelentősen gyorsabb, mint a legtöbb más fordított nyelvé, még akkor is, ha a nulláról építkezünk.
A fordító sebességének három fő oka van.
Először is, minden importált csomagot explicit módon fel kell sorolni minden forrásfájl elején, így a fordítónak nem kell egy egész fájlt elolvasnia és feldolgoznia, hogy meghatározza a függőségeket.
Másodszor, egy csomag függőségei egy irányított aciklikus gráfot alkotnak, és mivel nincsenek ciklusok, a csomagok külön-külön és esetleg párhuzamosan fordíthatók.
Végül, egy lefordított Go csomag objektumfájlja nemcsak magára a csomagra, hanem a függőségeire vonatkozó exportinformációkat is rögzíti.
Egy csomag fordításakor a fordítónak minden importáláshoz be kell olvasnia egy objektumfájlt, de ezeken a fájlokon túl nem kell néznie \cite{Alan15}.

\section{Tesztelés}
A mai programok természetesen sokkal nagyobbak és összetettebbek, emiatt rengeteg erőfeszítést tettek olyan technikák kifejlesztésére, amelyek ezt a komplexitást kezelhetővé teszik.
Különösen két technika emelkedik ki hatékonyságával.
Az első a programok rutinszerű szakértői felülvizsgálata, mielőtt bevezetésre kerülnének.
A második, amely ennek a fejezetnek a tárgya, a tesztelés.
A tesztelés, amely alatt implicit módon az automatizált tesztelést értjük, olyan kis programok írásának gyakorlata, amelyek ellenőrzik, hogy a tesztelt kód (a production kód) a várt módon viselkedik-e bizonyos bemenetek esetén, amelyek általában vagy gondosan megválasztottak, hogy bizonyos funkciókat gyakoroljanak, vagy véletlenszerűek, hogy széleskörű lefedettséget biztosítsanak.
A szoftvertesztelés területe hatalmas. A tesztelés feladata minden programozót időnként, néhány programozót pedig állandóan foglalkoztat.
A teszteléssel kapcsolatos szakirodalom több ezer nyomtatott könyvet és több millió szavas blogbejegyzéseket tartalmaz.
Minden elterjedt programozási nyelvben több tucat tesztkészítésre szánt szoftvercsomag létezik, némelyikben rengeteg elméletet, és úgy tűnik, a terület nem kevés prófétát vonz, akiknek kultikus követői vannak.
Ez szinte elég ahhoz, hogy meggyőzzük a programozókat arról, hogy a hatékony tesztek írásához egy egész sor új készséget kell elsajátítaniuk.
A Go tesztelési megközelítése ehhez képest meglehetősen alacsony technológiai színvonalúnak tűnhet.
Egyetlen parancsra támaszkodik, a go test , és egy sor konvencióra a tesztfüggvények írásához, amelyeket a go test képes lefuttatni.
A viszonylag könnyű mechanizmus hatékony a tiszta teszteléshez, és természetesen kiterjeszthető a benchmarkokra és a dokumentációhoz szükséges szisztematikus példákra.

A gyakorlatban a tesztkód írása nem sokban különbözik az eredeti program írásától. Rövid függvényeket írunk, amelyek a feladat egy részére összpontosítanak.
Vigyáznunk kell a peremfeltételekre, gondolkodnunk kell az adatszerkezeteken, és érvelnünk kell, hogy a megfelelő bemenetekből milyen eredményeket kell produkálnia a számításnak. De ez ugyanaz a folyamat, mint a közönséges Go kód írása; nem kell új jelöléseket, konvenciókat és eszközöket használni hozzá.
A go test alparancs a Go csomagok tesztelője, amelyek bizonyos konvenciók szerint vannak megszervezve. Egy csomagkönyvtárban a \_test.go végződésű fájlok nem részei a go build által épített csomagnak, de a go test által épített csomagnak igen.
A *\_test.go fájlokon belül háromféle funkciót kezelünk speciálisan : tesztek, benchmarkok és példák.
A tesztfüggvény, amely egy olyan függvény, amelynek neve Test -el kezdődik, a program logikájának helyes viselkedését vizsgálja; a go test meghívja a tesztfüggvényt és az eredményt jelenti, amely vagy "PASS" vagy "FAIL".
A benchmark függvény neve Benchmarkkal kezdődik, és valamilyen művelet teljesítményét méri; a go test a művelet átlagos végrehajtási idejét jelenti. Egy példafüggvény pedig, amelynek neve Example-vel kezdődik, gépileg ellenőrzött dokumentációt biztosít.
A go test eszköz átvizsgálja a *\_test.go fájlokat ezen speciális függvények után, létrehoz egy ideiglenes főcsomagot, amely mindegyiket a megfelelő módon hívja meg, felépíti és lefuttatja, jelenti az eredményeket, majd takarít \cite{Alan15}.