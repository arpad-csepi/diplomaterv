%----------------------------------------------------------------------------
\chapter{\clouds}
%----------------------------------------------------------------------------
A felhőalapú számítástechnika (későbbiekben cloud-computing) a számítástechnikai erőforrások - beleértve a tárolást, a feldolgozási teljesítményt, az adatbázisokat, a hálózatépítést, az analitikát, a mesterséges intelligenciát és a szoftveralkalmazásokat - interneten (a felhőben) történő rendelkezésre bocsátása.
Ezen erőforrások kiszervezésével a vállalatok hozzáférhetnek a számítási eszközökhöz, amelyekre szükségük van, amikor szükségük van rájuk, anélkül, hogy fizikai, helyben lévő IT-infrastruktúrát kellene vásárolniuk és fenntartaniuk.
Ez rugalmas erőforrásokat, gyorsabb innovációt és méretgazdaságosságot biztosít.
Sok vállalat számára a felhőre való áttérés közvetlenül kapcsolódik az adatok és az informatika modernizálásához \cite{cloudComp}.

\section{A felhőalapú számítástechnika jellemzői}
A cloud-computing előtt a szervezetek helyben vásároltak és tartottak fenn informatikai infrastruktúrát.
Bár a felhőre való kezdeti áttérés nagy részét a költségmegtakarítás vezérelte, sok szervezet úgy találja, hogy a nyilvános, a privát vagy a hibrid felhőinfrastruktúra számos előnyt kínál.  

Az agilis és DevOps-csapatok számára a felhőalapú számítástechnika lehetőséget nyújt a fejlesztési folyamat egyszerűsítésére és felgyorsítására. 

A felhőalapú számítástechnikát meghatározó jellemzői: \cite{cloudComp}.

\begin{itemize}
    \item Igény szerinti önkiszolgálás.
    \item Széles körű hálózati hozzáférés.
    \item Erőforrás-összevonás.
    \item Gyors rugalmasság.
    \item Mérhető szolgáltatás.
\end{itemize}

\section{A felhőalapú telepítések típusai}
A felhőalapú telepítéseknek három különböző típusa létezik.
Mindegyiknek megvannak az előnyei és a szervezetek gyakran többféle megoldás használatából is profitálnak \cite{cloudComp}.

\subsection{Publikus felhő}
A publikus felhők számítástechnikai erőforrásokat - szervereket, tárolókat, alkalmazásokat stb. -- nyújtanak az interneten keresztül egy felhőszolgáltatótól, például az AWS-től és a Microsoft Azure-tól.
A felhőszolgáltatók birtokolják és üzemeltetik az összes hardvert, szoftvert és egyéb támogató infrastruktúrát \cite{cloudComp}.

\subsection{Privát felhő}
A privát felhő olyan számítási erőforrások, amelyek kizárólag egy szervezet számára vannak fenntartva.
Fizikailag elhelyezkedhet a szervezet helyszíni adatközpontjában, vagy egy felhőszolgáltató által üzemeltetett tárhelyen.
A privát felhő a nyilvános felhőknél magasabb szintű biztonságot és adatvédelmet nyújt azáltal, hogy dedikált erőforrásokat kínál a vállalatok számára.

A privát felhő ügyfelei megkapják a nyilvános felhő elsődleges előnyeit, beleértve az önkiszolgálást, a skálázhatóságot és a rugalmasságot, de további előnyökkel, a további ellenőrzéssel és testreszabhatósággal.
Ráadásul a privát felhők magasabb szintű biztonsággal és adatvédelemmel rendelkezhetnek, mivel a nyilvános forgalom számára nem hozzáférhető magánhálózatokon vannak elhelyezve \cite{cloudComp}.

\subsection{Hibrid felhő}
A hibrid felhők a privát és a nyilvános felhők (például a Red Hat által támogatott IBM Hybrid Cloud) kombinációja, amelyeket olyan technológiával kapcsolnak össze, amely lehetővé teszi az adatok és az alkalmazások együttes működését.
Az érzékeny szolgáltatásokat és alkalmazásokat a biztonságos privát felhőben lehet tartani, míg a nyilvánosan elérhető webkiszolgálók és az ügyfelekkel szembenéző végpontok a nyilvános felhőben élhetnek.
A legtöbb népszerű harmadik féltől származó felhőszolgáltató hibrid felhőmodellt kínál, amely lehetővé teszi a felhasználók számára, hogy igényeiknek megfelelően kombinálják a privát és a nyilvános felhőket.
Ez nagyobb rugalmasságot biztosít a vállalkozások számára az alkalmazásuk egyedi infrastrukturális követelményeinek telepítésében \cite{cloudComp}.

\section{Felhőalapú számítástechnikai szolgáltatások}
A felhőalapú számítástechnika dinamikus tulajdonságai új, magasabb szintű szolgáltatások alapját képezik.
Ezek a szolgáltatások nemcsak kiegészíthetik, hanem gyakran szükséges szolgáltatásokat is nyújthatnak az agilis és DevOps-csapatok számára.

\subsection{Infrastruktúra mint szolgáltatás}
Az infrastruktúra mint szolgáltatás (későbbiekben Infrastucture as a Service vagy IaaS) egy olyan alapvető felhőszolgáltatási réteg, amely lehetővé teszi a szervezetek számára, hogy informatikai infrastruktúrát - szervereket, tárolókat, hálózatokat, operációs rendszereket - béreljenek egy felhőszolgáltatótól.
Az IaaS lehetővé teszi a felhasználók számára, hogy a nyers fizikai szerverraktárakból lefoglalják és rendelkezésre bocsássák a szükséges erőforrásokat. Ezenkívül az IaaS lehetővé teszi a felhasználók számára, hogy előre konfigurált gépeket foglaljanak speciális feladatokra, például terheléselosztókra, adatbázisokra, e-mail szerverekre, elosztott várólistákra.

A DevOps-csapatok az IaaS-t olyan mögöttes platformként használhatják, amelyre egy DevOps-eszközláncot építhetnek, amely magában foglalhatja különböző harmadik féltől származó eszközök használatát \cite{cloudComp}.

\subsection{Platform mint szolgáltatás}
A platform mint szolgáltatás (későbbiekben Platform as a Service vagy PaaS) egy IaaS-re épülő felhőinfrastruktúra, amely erőforrásokat biztosít felhasználói szintű eszközök és alkalmazások létrehozásához.
Az alapul szolgáló infrastruktúrát, beleértve a számítási, hálózati és tárolási erőforrásokat, valamint a fejlesztőeszközöket, az adatbázis-kezelő rendszereket és a köztes szoftvereket \cite{cloudComp}.

A PaaS kihasználja az IaaS-t, hogy automatikusan kiossza a nyelvi alapú tech stack működtetéséhez szükséges erőforrásokat.
Népszerű nyelvi tech stackek a Ruby On Rails, a Java Spring MVC, a MEAN és a JAM stackek. A PaaS-ügyfelek ezután egyszerűen feltölthetik az alkalmazáskódjukat, amely automatikusan települ a PaaS infrastruktúrájára.
Ez egy újszerű és hatékony munkafolyamat, amely lehetővé teszi a csapatok számára, hogy teljes mértékben az adott üzleti alkalmazás kódjára koncentráljanak, és ne aggódjanak a tárhely és az infrastruktúra problémái miatt.
A PaaS automatikusan kezeli az infrastruktúra skálázását és felügyeletét, hogy a megfigyelt forgalmi terhelésnek megfelelően növelje vagy csökkentse az erőforrásokat \cite{cloudComp}.

\subsection{Szoftver mint szolgáltatás}
A szoftver mint szolgáltatás (későbbiekben Software as a Service vagy SaaS) az interneten keresztül, igény szerint és jellemzően előfizetés alapján nyújt szoftveralkalmazásokat.
A felhőszolgáltatók üzemeltetik és kezelik az alkalmazást, és szükség szerint gondoskodnak a szoftverfrissítésekről és a biztonsági javításokról.
A SaaS-re példák a CRM-rendszerek, webmail-alkalmazások, termelékenységi eszközök, mint a Jira és a Confluence, elemzőeszközök, felügyeleti eszközök, csevegőalkalmazások és még sok más \cite{cloudComp}.

\subsection{Funkció mint szolgáltatás}
A szolgáltatásként nyújtott funkció (Function as a Service vagy FaaS) olyan felhőalapú számítástechnikai szolgáltatás, amely olyan platformot kínál, amelyen az ügyfelek alkalmazásokat fejleszthetnek, futtathatnak és kezelhetnek.
Ezáltal a fejlesztőknek nem kell kiépíteniük és fenntartaniuk az alkalmazás fejlesztéséhez és elindításához szükséges infrastruktúrát.
A felhőszolgáltatók felhőalapú erőforrásokat kínálnak, végrehajtanak egy kódblokkot, visszaadják az eredményt, majd megsemmisítik a felhasznált erőforrásokat \cite{cloudComp}.

\section{A felhőalapú számítástechnika előnyei}
\subsection*{Csökkentett költség}
A felhőalapú erőforrásokat használó csapatoknak nem kell saját hardvereszközöket vásárolniuk.
A hardverköltségeken túl a felhőszolgáltatók mindent megtesznek a hardverhasználat maximalizálása és optimalizálása érdekében.
Ezáltal a hardver és a számítási erőforrások árucikké válnak, és a felhőszolgáltatók a legalacsonyabb végeredményért versenyeznek \cite{cloudComp}.

\subsection*{Nagyobb skálázhatóság}
Mivel a felhőalapú számítástechnika alapértelmezés szerint rugalmas, a szervezetek igény szerint méretezhetik az erőforrásokat.
A felhőalapú számítástechnika automatikus skálázási funkciókat tesz lehetővé a csapatok számára.
A felhőalkalmazások a forgalom kiugrásaira reagálva automatikusan képesek csökkenteni és növelni infrastrukturális erőforrásaikat \cite{cloudComp}.

\subsection*{Nagyobb teljesítmény}
A felhőalapú számítástechnika a legújabb és legnagyobb számítási erőforrásokat kínálja.
A felhasználók hozzáférhetnek a legújabb, extrém, többmagos CPU-kkal felszerelt, nehéz párhuzamos feldolgozási feladatokra tervezett gépekhez.
Emellett a nagy felhőszolgáltatók élvonalbeli GPU és TPU hardveres gépeket kínálnak az intenzív grafikai, mátrix- és mesterséges intelligencia feldolgozási feladatokhoz.
Ezek a felhőszolgáltatók folyamatosan frissítik a legújabb processzortechnológiát.

A nagyobb felhőszolgáltatók globálisan elosztott hardverhelyekkel rendelkeznek, amelyek a fizikai kapcsolódási helytől függően nagy teljesítményű kapcsolatokat biztosítanak.
Emellett a felhőszolgáltatók globális tartalomszolgáltató hálózatokat kínálnak, amelyek a felhasználói kéréseket és tartalmakat hely szerint gyorsítótárba helyezik \cite{cloudComp}.

\subsection*{Nagyobb végrehajtási sebesség}
A felhőinfrastruktúrákat használó csapatok gyorsabban tudnak végrehajtani és értéket nyújtani ügyfeleiknek.
Az agilis szoftvercsapatok kihasználhatják a felhőinfrastruktúrát, hogy gyorsan új virtuális gépeket indíthassanak az egyedi ötletek kikísérletezéséhez és validálásához, valamint automatizálhatják a pipeline tesztelési és telepítési fázisait \cite{cloudComp}.

\subsection*{Nagyobb biztonság}
A privát felhő tárhely elszigetelt tűzfalas infrastruktúrát kínál, amely javítja a biztonságot.
Emellett a felhőszolgáltatók számos biztonsági mechanizmust és technológiát kínálnak a biztonságos alkalmazások kialakításához.
A felhasználók hozzáférésének ellenőrzése fontos biztonsági szempont, és a legtöbb felhőszolgáltató kínál eszközöket a felhasználók hozzáférésének részletes korlátozására \cite{cloudComp}.

\subsection*{Folyamatos integráció és szállítás}
A folyamatos integráció és folyamatos szállítás (későbbiekben CI/CD) a DevOps-gyakorlók egyik legfontosabb gyakorlata, amely segít növelni a csapat sebességét és csökkenteni a piacra jutási időt.
A felhőalapú CI/CD, például a Bitbucket Pipelines lehetővé teszi a csapatok számára, hogy automatikusan építsenek, teszteljenek és telepítsenek kódot anélkül, hogy a CI-infrastruktúra kezelésével vagy karbantartásával kellene foglalkozniuk.
A Bitbucket Pipelines a Docker konténerekre támaszkodik a kiadási csővezeték elszigeteltségének és reprodukálhatóságának biztosítása érdekében.
A csapatok hasonló parancsokat futtathatnak, mint egy helyi gépen, de minden egyes buildhez friss és reprodukálható beállítás minden előnyével \cite{cloudComp}.

\subsection*{Átfogó felügyelet és incidenskezelés}
A felhőalapú telepítések lehetővé teszik a csapatok számára, hogy eszközeiket a végponttól a végpontig összekapcsolják, megkönnyítve a csővezeték minden részének nyomon követését.
Az átfogó felügyelet egy másik kulcsfontosságú képesség a DevOps-ot gyakorló szervezetek számára, mivel lehetővé teszi a problémák és incidensek gyorsabb kezelését.
A felhőszolgáltatók megosztják a rendszer állapotára vonatkozó mérőszámokat, beleértve az alkalmazás és a szerver CPU-ját, memóriáját, a kérések számát, a hibaarányt, az átlagos válaszidőt stb. Például a terhelés figyelése sok virtuális gépen keresztül azt jelenti, hogy a csapatok több kapacitást tudnak hozzáadni, ha megnövekszik az igény, vagy a csapatok automatizálhatják a skálázást (fel/le) ezen mérőszámok alapján, hogy csökkentsék az emberi beavatkozást és a költségeket \cite{cloudComp}.