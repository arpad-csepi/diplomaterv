%----------------------------------------------------------------------------
\chapter{\summary}
%----------------------------------------------------------------------------
Egyetlen Kubernetes klaszter számos opciót és lehetőséget kínál.
Azonban ha nagyobb rugalmasságot, skálázhatóságot, ellenállóképességet és biztonságot keresünk, akkor a több klaszteres architektúra jobb választás az alkalmazásunk számára.
Ez nem jelenti azt, hogy a több klaszteres Kubernetes megoldásnak nincsen hátulütője.
A vele járó előnyök ellenére van néhány komoly kihívás is, amit figyelembe kell venni.
Ezekből a kihívásokból igyekeztem lefaragni a programom elkészítésével, mely felhasználóbarát jellegének köszönhetően több ember érdeklődését is felkelti.
Egy ilyen program megírása viszont nem kis feladat.
Sok technológia értése szükséges ahhoz, hogy hatékonyan lehessen megírni és használni egy ilyen programot.

A dolgozatomban először megismerkedtem ezekkel a sokszínű technológiákkal.
A Kubernetes területe gyorsan képes változni, így nem volt egyszerű megtalálni minden forrást elsőre.

Ezután elkeztem tervezni a programom felépítését, melynek voltak meghatározott céljai.
Ezek a célok segítettek abban, hogy fókuszálni tudjak a lényeges funkciókra, melyeket fejleszteni szerettem volna.
A Cobra CLI csomag segítségével képes voltam egy vázat generálni, melyre tudtam építkezni.

A szakdolgozat elkészítésével rengeteget tanultam a Kubernetes-ről, az Istio-ról, a Helm-ről és a Go programozásról.
Szerencsére voltak ismereteim a Linux rendszerek műküdéséről, Docker használatáról, C és Python programozásról és a GitHub-ról is, így nem féltem belevágni ebbe a témába.
Számomra nagyon érdekes volt a téma és örülök, hogy megtapasztalhattam a csapatmunkában való részvételt is.
Ezekkel és más kapcsolódó technológiákkal később is fogok foglalkozni, bővíteni még jobban a tudásomat.